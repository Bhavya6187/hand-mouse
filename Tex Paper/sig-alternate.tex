% This is "sig-alternate.tex" V1.9 April 2009
% This file should be compiled with V2.4 of "sig-alternate.cls" April 2009
%
% This example file demonstrates the use of the 'sig-alternate.cls'
% V2.4 LaTeX2e document class file. It is for those submitting
% articles to ACM Conference Proceedings WHO DO NOT WISH TO
% STRICTLY ADHERE TO THE SIGS (PUBS-BOARD-ENDORSED) STYLE.
% The 'sig-alternate.cls' file will produce a similar-looking,
% albeit, 'tighter' paper resulting in, invariably, fewer pages.
%
% ----------------------------------------------------------------------------------------------------------------
% This .tex file (and associated .cls V2.4) produces:
%       1) The Permission Statement
%       2) The Conference (location) Info information
%       3) The Copyright Line with ACM data
%       4) NO page numbers
%
% as against the acm_proc_article-sp.cls file which
% DOES NOT produce 1) thru' 3) above.
%
% Using 'sig-alternate.cls' you have control, however, from within
% the source .tex file, over both the CopyrightYear
% (defaulted to 200X) and the ACM Copyright Data
% (defaulted to X-XXXXX-XX-X/XX/XX).
% e.g.
% \CopyrightYear{2007} will cause 2007 to appear in the copyright line.
% \crdata{0-12345-67-8/90/12} will cause 0-12345-67-8/90/12 to appear in the copyright line.
%
% ---------------------------------------------------------------------------------------------------------------
% This .tex source is an example which *does* use
% the .bib file (from which the .bbl file % is produced).
% REMEMBER HOWEVER: After having produced the .bbl file,
% and prior to final submission, you *NEED* to 'insert'
% your .bbl file into your source .tex file so as to provide
% ONE 'self-contained' source file.
%
% ================= IF YOU HAVE QUESTIONS =======================
% Questions regarding the SIGS styles, SIGS policies and
% procedures, Conferences etc. should be sent to
% Adrienne Griscti (griscti@acm.org)
%
% Technical questions _only_ to
% Gerald Murray (murray@hq.acm.org)
% ===============================================================
%
% For tracking purposes - this is V1.9 - April 2009

\documentclass{sig-alternate}

\begin{document}
%
% --- Author Metadata here ---
\conferenceinfo{IITM-2010}{International Conference on Intelligent Interactive Technologies and Multimedia}
%\CopyrightYear{2007} % Allows default copyright year (20XX) to be over-ridden - IF NEED BE.
%\crdata{0-12345-67-8/90/01}  % Allows default copyright data (0-89791-88-6/97/05) to be over-ridden - IF NEED BE.
% --- End of Author Metadata ---

\title{Implementation of Hand Mouse : Hand Segmentation and Optical Flow}

%
% You need the command \numberofauthors to handle the 'placement
% and alignment' of the authors beneath the title.
%
% For aesthetic reasons, we recommend 'three authors at a time'
% i.e. three 'name/affiliation blocks' be placed beneath the title.
%
% NOTE: You are NOT restricted in how many 'rows' of
% "name/affiliations" may appear. We just ask that you restrict
% the number of 'columns' to three.
%
% Because of the available 'opening page real-estate'
% we ask you to refrain from putting more than six authors
% (two rows with three columns) beneath the article title.
% More than six makes the first-page appear very cluttered indeed.
%
% Use the \alignauthor commands to handle the names
% and affiliations for an 'aesthetic maximum' of six authors.
% Add names, affiliations, addresses for
% the seventh etc. author(s) as the argument for the
% \additionalauthors command.
% These 'additional authors' will be output/set for you
% without further effort on your part as the last section in
% the body of your article BEFORE References or any Appendices.

\numberofauthors{4} %  in this sample file, there are a *total*
% of EIGHT authors. SIX appear on the 'first-page' (for formatting
% reasons) and the remaining two appear in the \additionalauthors section.
%
\author{
% You can go ahead and credit any number of authors here,
% e.g. one 'row of three' or two rows (consisting of one row of three
% and a second row of one, two or three).
%
% The command \alignauthor (no curly braces needed) should
% precede each author name, affiliation/snail-mail address and
% e-mail address. Additionally, tag each line of
% affiliation/address with \affaddr, and tag the
% e-mail address with \email.
%
% 1st. author
\alignauthor
Bhavya Agarwal\\
       \affaddr{The LNM Institute of Information Technology}\\
       %\affaddr{1932 Wallamaloo Lane}\\
       \affaddr{Jaipur, India}\\
       \email{trovato@corporation.com}
%\and  % use '\and' if you need 'another row' of author names
% 2nd. author
\alignauthor
Chirag Jain\\
       \affaddr{The LNM Institute of Information Technology}\\
       %\affaddr{1932 Wallamaloo Lane}\\
       \affaddr{Jaipur, India}\\
       \email{trovato@corporation.com}
\and 
% 3rd. author
\alignauthor Kumar Harsh Srivastava\\
        \affaddr{The LNM Institute of Information Technology}\\
       %\affaddr{1932 Wallamaloo Lane}\\
       \affaddr{Jaipur, India}\\
       \email{trovato@corporation.com}
% 4th. author
\alignauthor Manohar Kuse\\
     \affaddr{The LNM Institute of Information Technology}\\
       %\affaddr{1932 Wallamaloo Lane}\\
       \affaddr{Jaipur, India}\\
       \email{trovato@corporation.com}
}
% There's nothing stopping you putting the seventh, eighth, etc.
% author on the opening page (as the 'third row') but we ask,
% for aesthetic reasons that you place these 'additional authors'
% in the \additional authors block, viz.

\date{30 July 1999}
% Just remember to make sure that the TOTAL number of authors
% is the number that will appear on the first page PLUS the
% number that will appear in the \additionalauthors section.

\maketitle
\begin{abstract}
This paper describes the algorithm used for implementation of a hand mouse. Hand segmentation
and tracking was done using the output obtained with an over-head camera. Hand segmentation
was a two step process, which involved thresholding in HSV space and then filtering the blobs 
based on size. This biggest blob was then passed as an input to the Shi and Pomasi algorithm 
which produced the best tracking points inside the blob.
The obtained points were tracked with Lucas-Kanade algorithm. The velocity obtained \cite{bowman:reasoning}
by using these tracked pixels were served as inputs to the mouse movement, which was scaled by the screen resolution.
\end{abstract}

% A category with the (minimum) three required fields
%\category{H.4}{Information Systems Applications}{Miscellaneous}
%A category including the fourth, optional field follows...
%\category{D.2.8}{Software Engineering}{Metrics}[complexity measures, performance measures]

%\terms{Theory}

\keywords{Hand Mouse, Optical Flow, Tracking, HSV Thresholding}

\section{Introduction}
A creature either biological or mechanical is to interact effectively with its environment,  it
needs to know what objects are where. Computer vision provides a primary method for understanding 
how to make intelligent decisions about an environment, on the basis of sensory inputs.
In this modern world of changing times, display devices, cameras, mobile phones are getting more 
and more interactive and intuitive. Such technologies are are expected to grow in the future.
Computer vision based input devices and sensors are becoming prominent owing to faster 
processing power that now available in modern day computers. This increased computing speed 
has help in development of application which can run in real-time environment with modest
reqirements. The up-coming fields like optical flow provide an excellent opportunity to design such 
application with robustness and speed. 

In this paper, we describe the details of the algorithms and the implementation of the hand mouse
that we have implemented. A hand mouse is a concept, of tracking your hand's motion and 
moves your mouse pointer the screen accordingly. It can be used as a deviceless mouse. 
Only device that is needed is a overhead web-cam. It provides better control and sensitivity 
option than a normal mouse. We have prototyped our application using the Intel OpenCV library for
image processing and X11 library for the mouse movements on the screen.

The basic procedure that is followed is that, optical flow is computed on a specified 
region which was found with HSV based thresholding and connected component analysis.
The movements thus obtained was passed on to the X11 library functions and scaled appropriately
for the mouse movements. The prespecified color strip was placed on the hand, which helped in
determining the points which were to be tracked. These points were tracked with the Pyramidal Lucas-Kanade 
algorithm\cite{lucas81}.

% The following two commands are all you need in the
% initial runs of your .tex file to
% produce the bibliography for the citations in your paper.
\bibliographystyle{abbrv}
\bibliography{sigproc}  % sigproc.bib is the name of the Bibliography in this case
% You must have a proper ".bib" file
%  and remember to run:
% latex bibtex latex latex
% to resolve all references
%
% ACM needs 'a single self-contained file'!
%
%APPENDICES are optional
%\balancecolumns
% That's all folks!
\end{document}
